\begin{recipe}
[ %
	preparationtime = {\SI{25}{\minute}},
%	bakingtime={\SI{1}{\hour}},
%	bakingtemperature={\protect\bakingtemperature{topbottomheat=\SI{280}{\celsius}}},
	portion = {\portion[Servings (Serving size: 1 fillet and about \SI{1/2}{\cup} salsa)]{4}},
	source = {HPBraincase},
	sourceref = {\href{https://www.myrecipes.com/recipe/grilled-king-salmon-tomato-peach-salsa}{MyRecipes}}
    ]{Grilled Salmon with Tomato-Peach Salsa}


    \introduction{
		Use a peach that's just ripe so it's juicy but still holds its shape. King/chinook salmon is the best quality and works well in this dish, though sockeye works.
	}

	\ingredients[]{
		\SI{1}{\cup} & Chopped peeled peach \\
		\SI{3/4}{\cup} & Quartered cherry tomatoes \\
		\SI{1/4}{\cup} & Thinly vertically sliced red onion \\
		\SI{3}{\tablespoon} & Small fresh mint leaves \\
		\SI{3}{\tablespoon} & Small fresh basil leaves \\
		\SI{2}{\tablespoon} & Fresh lemon juice \\
		\SI{1}{\tablespoon} & Extra-virgin olive oil \\
		\SI{1}{\tablespoon} & Honey \\
		1 & Jalape\~no pepper, thinly sliced (optional) \\
		\SI{1}{\teaspoon} & Kosher salt, divided \\
		4 (6-ounce) & Wild Alaskan king salmon fillets \\
		\SI{1/4}{\teaspoon} & Freshly ground black pepper \\
		& Cooking spray
	}

	\preparation{
		\step Preheat grill to high heat.

		\step Combine cherry tomatoes, onion, mint, basil, lemon juice, olive oil, and honey in a bowl. Add jalape\~no, if desired.

		\step Season mixture with \SI{1/4}{\teaspoon} salt, then toss gently.

		\step Season fillets with remaining \SI{3/4}{\teaspoon} salt and black pepper.

		\step Coat a grill rack with cooking spray. Place the fillets on the grill rack, and grill for \SI{10}{\minute} or until desired degree of doneness, turning after \SI{5}{\minute}.

		\step Serve with salsa.
	}

\end{recipe}