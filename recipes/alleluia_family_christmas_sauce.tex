\begin{recipe}
[ %
	preparationtime = {\SI{10}{\hour}+},
%	bakingtime={\SI{1}{\hour}},
%	bakingtemperature={\protect\bakingtemperature{topbottomheat=\SI{280}{\celsius}}},
	portion = {\portion[Item]{1}},
	source = {BrainStew}
    ]{The Alleluia Family Christmas Sauce}

    \introduction{
    	This is a recipe that is made yearly, and has been for generations. Due to that history, there are no real measurements and there are many optional steps or choices. I have tried to preserve the spirit of the recipe as much as possible.
	}

	\ingredients[22]{
		$5 \times \SI{28}{\ounce}$ & Canned Whole Tomatoes, Ideally San Marzano with DOP mark \\
		\SIrange{1}{2}{cans} & Tomato Paste \\
		\SI{5}{cloves} & Garlic \\
		& Olive Oil (Good quality, but not great)  \\
		3 & Large Bay Leaves (dried) \\
		& Dried Italian Herbs (optional, which ones you choose are up to you) \\
		& Fresh Basil (optional, could use all fresh herbs as well) \\
		1 & Large Potato, peeled \\
		1 & Large Carrot, peeled \\
		1 & Large Onion, peeled \\
		& Meatballs, cooked (Ideally, the Alleluia family recipe) \\
		& Sausage, cooked (Ideally, both Hot and Sweet Italian Sausage)
	}

	\preparation{

		This is an optional starting step implemented by Xela and Brainstew.

		\step Add the tomato paste and a small amount of liquid from the tomatoes into a small separate pan and caramelize until it is almost burnt.

		This adds a huge depth of flavor that mimics letting the sauce itself get nearly burnt to bring out the sugars in the tomato sauce, but has a bit less risk of actually burning the sauce.

		\step Dice, smash, or otherwise pulverize at least \SI{5}{cloves} of garlic. Cover the bottom of your large sauce pot with oil, and fry the garlic until just golden and fragrant.

		While the garlic is frying, break down the tomatoes either using a potato masher or your hands until they are not whole, but still chunky. They will break down further as the sauce cooks.

		\step Once the garlic is fried, carefully add the tomatoes (the oil may splatter at you), potato, carrot, onion, bay leaves, and any fresh herbs you may be using.

		The potato here is used to absorb acid, the onion to add onion flavor, and the carrot for sweetness.

		\step Reduce (stirring occasionally as needed, depending on heat) until the sauce has lost a couple of inches of liquid and the flavor of the onion and fresh herbs have infused completely into the sauce.

		You may need to stir occasionally depending on how high you make the heat, and the higher you cook it the faster this process will go.

		\step \label{step:once}Once the onion flavor is coming through, add any dried herbs you would like.

		\vspace{1em}

		\step \label{step:soup} At this point the sauce should be like a thick soup. Add in your tomato paste (raw or pre-caramelized) which will further thicken the sauce.

		Continue reducing until the sauce has thickened to just about your desired consistency. The sauce will continue to reduce, so it is okay if it still is a bit loose. Make sure the tomato paste has lost its raw, tart flavor.

		Between Steps~\ref{step:once} and \ref{step:soup}, you will likely be reducing the sauce in this phase for around 5 hours.

		\step Fish out the carrot, onion, potato, and any whole herbs. If you are not a meat-person, the sauce is now complete.

		\step If you are making this with sausage or meatballs, you will now need to add in the meat. You should end up with enough space within the sauce to just barely stir the sauce without breaking up the meatballs and sausage.

		Turn the heat down to low, and keep a tight eye on the sauce as it continues to cook. Stir the sauce occasionally to avoid burning, and make sure you are careful not to crush any of the meatballs.

		This process can take as long as you are willing to give it. The longer you cook, the more meaty complex flavor the sauce will have. I suggest you allow the sauce to cook for as long as you have time.

		This step is likely to also take around \SI{5}{\hour} or longer.

		\step Remove the meatballs and sausage into a separate container.

		\vspace{1em}

		Serve the sauce over manicotti or other pasta dishes with the meatballs and sausage on the side.

	}

\end{recipe}