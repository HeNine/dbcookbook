\begin{recipe}
[ %
	preparationtime = {\SI{1}{\hour}},
	bakingtime={\SIrange{20}{24}{\minute}},
	bakingtemperature={\protect\bakingtemperature{topbottomheat=\SI{350}{\fahrenheit} (\SI{175}{\celsius})}},
	portion = {\portion[Muffins]{18--20}},
	source = {chimingfish}
    ]{Apple Muffins}
    
    \introduction{
    	I usually cut down the granulated sugar to \SI{1.5}{cups}. 
    	
    	Any baking apple will do here, although Granny Smith is probably a good standby? 
    	
    	Also, I'd say that a mixer isn't strictly necessary here, although it is nice if you have one.
	}

	\ingredients[13]{
		\SI{2}{cups} & Granulated sugar (1.5 for a less sweet muffin) \\
		2 & Large eggs \\
		\SI{1}{\cup} & Vegetable oil \\
		\SI{1}{\tablespoon} & Vanilla extract \\
		\SI{3}{cups} & All-purpose flour \\
		\SI{1}{\teaspoon} & Salt \\
		\SI{1}{\teaspoon} & Baking soda \\
		\SI{1}{\teaspoon} & Ground cinnamon \\
		\SI{3}{cups} & Peeled, cored, diced apples (around 3 apples) \\
		& Brown sugar for topping (around \num{1/2} cup)
	}

	\preparation{
		\step Preheat oven to \SI{350}{\fahrenheit} (\SI{175}{\celsius}) and line muffin pan with 18--20 paper liners.
		
		\step With a mixer, cream together sugar, eggs, oil, and vanilla. The mixture should be a pale yellow.
		
		\step In a separate bowl, whisk together flour, baking soda, salt, and ground cinnamon. 
		
		\step Add dry ingredients to creamed mixture and mix until combined. The batter will be very thick almost like the texture of cookie dough. Mix in the diced apples. The dough will loosen up a bit when the apples are mixed in.
		
		\step Fill paper liners almost to the top, about \num{3/4} of the way full. Sprinkle each muffin top generously with brown sugar.
		
		\step Bake at \SI{350}{\fahrenheit} (\SI{175}{\celsius}) for \SIrange{20}{24}{\minute}.
	}
    
\end{recipe}