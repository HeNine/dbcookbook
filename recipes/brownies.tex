\begin{recipe}
[ %
	preparationtime = {\SI{45}{\minute}},
	bakingtime={\SI{30}{\minute}},
	bakingtemperature={\protect%
		\bakingtemperature{topbottomheat=\SI{180}{\celsius},
							fanoven=\SI{160}{\celsius}}
	},
	portion = {\portion[Brownies]{some}},
	source = {HeNine},
	sourceref = {Backen mit Liebe}
	]{Brownies}

	\ingredients[]{
		\SI{100}{\gram} & 70\% dark chocolate \\
		\SI{110}{\gram} & Butter \\
		\SI{125}{\gram} & Sugar \\
		\SI{100}{\gram} & Brown sugar \\
		1\,pinch & salt \\
		1\,tsp & Ground vanilla (or vanilla sugar) \\
		2 & Eggs \\
		1 & Yolk \\
		\SI{120}{\gram} & Flour \\
		3\,tbsp & Cocoa powder
	}

	\preparation{
		\step Preheat the oven to \SI{180}{\celsius}. Line $\SI{18}{\centi\meter}\times\SI{18}{\centi\meter}$ (or equivalent) pan with parchement paper.

		\step In a double boiler melt butter and chocolate, mix in both types of sugar, and the salt.

		Wait for the mixture to cool a bit so it doesn't cook the eggs.

		\step Mix in the vanilla, eggs and the yolk.

		\vspace{1em}

		\step Sift in the flour and cocoa powder. Stir just enough to combine all the ingredients.

		\step Pour the mixture into the pan and smooth it out. Bake for \SI{30}{\minute} or until a toothpick stuck in the middle comes out with moist crumbs.

		Unlike cakes, the brownies should still be a bit moist in the center, so take them out before they bake completely through. If the tootpick comes out with a smear of dough, though, give it a bit more time.

		\step Let the brownies cool for a bit, then turn them out onto a cooling rack.
		\vspace{1em}

		\step After they are completely cooled cut them into cuboids of desired size. It is recommended that you put them in the fridge overnight to let the moisture redistribute a bit.
	}
\end{recipe}