\begin{recipe}
[ %
	preparationtime = {\SI{3}{\hour}},
	bakingtime={\SIrange{40}{60}{\minute}},
	bakingtemperature={\protect\bakingtemperature{topbottomheat=~\SI{375}{\fahrenheit} (\SI{190}{\celsius})}},
	portion = {\portion{3--4}},
	source = {HubbeKing}
    ]{{\scripterfont Hubbe's Famous Lasagna}}

    \introduction{
    	The cheese sauce also makes for a VERY good mac'n'cheese, just combine the cheese sauce with an amount of partially cooked noodles, add to a suitably shaped dish, cook in oven, topped with some grated cheese and breadcrumbs. Good stuff.

    	Both the meat and cheese sauces can be used separately or combined as sauce for all sorts of stuff, like pasta or risottos. Cooking is like 40\% trying stuff and seeing if it works out.

    	In case of leftover sauces, combine them into one and store in a container. Reheat later, serve with pasta, or rice, or some other carbohydrate. It’s a good, cheesy sauce, Bront.

    	Sauce consistency is tricky to describe, but by trying different consistencies out you’ll start to figure out what works and what results in dry lasagna and what results in pasta soup.

    	\textbf{\large Cheese}

    	For cheese sauce use gouda, edam -- pick one you like, so long as it melts. For topping mozzarella works, but so does literally any other.

    	\textbf{Important:} Cold milk, for real. Cold milk + hot roux = no lumps. SCIENCE!

    	\textbf{\large Herbs}

    	Besides listed herbs, you can use other herbs, if you want. I’m not a cop. Chives are real good though.
    	\vspace{2in}
	}

	\ingredients[17]{
		\SIrange{1}{2}{packages} & Lasagna sheets\\
		\SI{1}{\pound} & Ground beef (lean) \\
		~\SI{1}{\pound} & Crushed tomatoes (\SIrange{1x1/2}{2}{cans}) [\SI{14}{\flounce} or so?] \\
		1 & Medium yellow onion OR 2 shallots \\
		\SIrange{2}{14}{cloves} & Garlic, to taste \\
		\SI{2}{\tablespoon} & Butter (preferably unsalted, but not a big deal) \\
		\SI{2}{\tablespoon} & All-purpose flour \\
		~\SIrange{1}{2}{cups} & Milk (cold) \\
		\SI{1/2}{\pound} & Semi-hard cheese \\
		& Additional cheese for topping \\
		& Salt \\
		& Pepper (ground) \\
		& Paprika \\
		& Allspice (ground) \\
		& Oregano (dried) \\
		& Basil (dried) \\
		& Soy sauce \\
		& Red wine OR Balsamic vinegar \\
		& Ground nutmeg
	}



	\preparation{

		\textbf{\large Creating meat sauce}

		\step Chop onion and garlic (crush the garlic if you want, that’s easier).

		\step Heat frying pan to medium-high.

		\vspace{1em}

		\step Add crushed tomatoes to medium-sized pot, along with \SIrange{1/2}{1}{can} of water. Place pot on stove; begin gentle simmer.

		\step Add frying lubricant of choice to hot frying pan (vegetable oil, olive oil, butter, you do you).

		\step Add onion to frying pan, gently fry until translucent.

		\step Add garlic and ground beef to frying pan. Fry until nicely browned.

		\step Spice the ground beef:

		\begin{itemize}
			\item Paprika, like lots. Couple of tablespoons?

			\item Salt, pepper, allspice. You’ll wanna add a little bit, let soak for a minute, taste, and adjust to taste.
		\end{itemize}


		\step Once happy with spices, splash in a few tablespoons of soy sauce for salt, umami, and also to deglaze pan a bit.

		\step Remove ground beef from frying pan, scrape into pot alongside tomato/water mixture.

		\step Add some red wine OR balsamic vinegar into sauce, along with herbs of choice, but be cautious. Some herbs have a surprising amount of taste to them when added into a sauce and could overpower the meat/tomato goodness (LOOKING AT YOU, OREGANO).

		Like a couple pinches to a teaspoon or so of each herb is probably fine though?

		\textbf{\large CONGRATULATIONS, you have meat sauce. Taste for seasoning.}

		\textbf{\large Creating cheese sauce}

		\step Grate cheese.

		\vspace{1em}

		\step Get out a whisk.

		\vspace{1em}

		\step Add a medium-sized pot onto the stove, crank heat onto high.

		\vspace{1em}

		\step Melt \SI{2}{\tablespoon} of butter into pot.

		\vspace{1em}

		\step Once molten, add \SI{2}{\tablespoon} of all-purpose flour. Stir gently for a minute, on high heat.

		\step Begin adding COLD milk, little by little.

		\vspace{1em}

		\textbf{\Huge DO NOT. STOP. STIRRING.}

		The milk will HAPPILY burn quite quickly if you do.

		\step Once you’ve added \SI{2}{cups} of milk, or the cheese sauce reaches a nice consistency, turn heat down to low.

		\step Gently simmer milk sauce for a few minutes.

		\vspace{1em}

		\step Take sauce off heat. Add some spices.

		\begin{itemize}
			\item Salt and pepper, a couple pinches. Use more salt if you used unsalted butter.

			\item Pinch or two of nutmeg. No more. Too much and the nutmeg will overpower everything in the lasagna and you will be left with a cheesy nutmeg pasta cake.
		\end{itemize}


		\step Add your grated cheese and stir with a whisk until sauce has no lumpy bits.

		\vspace{1em}

		\step If the cheese sauce is too thick, add more cold milk, return to low heat and bring back up to a simmer -- if the cheese sauce is too thin, simmer for a few additional minutes.

		\textbf{\large CONGRATULATIONS -- cheese sauce has occurred. Taste for seasoning.}

		\clearpage

		\textbf{\large CONSTRUCTING LASAGNA}

		\step Heat oven to ~\SI{375}{\fahrenheit} (\SI{190}{\celsius}).

		\vspace{1em}

		You know how this works, right? Layering sauce and sheets. Pretty straight-forward stuff.

		\step Begin with a bottom layer of meat sauce in your chosen pan. This is to help the lasagna not stick to the pan. You can also grease the pan if you want, but that will make the final product more greasy.

		\step If in doubt, use more sauce. Saucier lasagna is moister lasagna is more delicious lasagna.

		\step Cover meat sauce layer with lasagna sheets, and cover those sheets in cheese sauce.

		\step Cover cheese sauce layer with lasagna sheets, and cover those sheets in meat sauce.

		For additional structural integrity, alternate directions for the lasagna sheets by \ang{90} each layer

		\step Continue until one of the following occurs:

		\begin{enumerate}
			\item You run out of sheets.

			\item You run out of one of the sauces.

			\item You run out of space in the pan.
		\end{enumerate}


		Make sure you have enough cheese or meat sauce to properly cover the top layer of sheets (if you wanna be FANCY you can use BOTH SAUCES on top).

		\step Cover the top sauce layer in cheese.

		\vspace{1em}

		\step Insert completed lasagna into hot oven, approximately in the middle of the oven.

		\step Wait \SIrange{40}{60}{\minute}, until lasagna is bubbly and the cheese on top is nicely browned.

		\vspace{1em}

		Depending on what exact lasagna noodles you’re using, you may need more or less time. Some noodles need lots of time to not be chewy, some need real little.

		Consult packaging for noodles, it usually knows what it’s talking about.

		For longer cooking time, adjust temperature downwards and/or move lasagna lower in the oven to prevent the top cheese layer from burning

		\step Remove lasagna from oven, and cover with foil.

		\vspace{1em}

		\step Wait \SIrange{15}{30}{\minute} before slicing and serving.

		\vspace{1em}

		\step Eat with chosen delicious drink, optionally ketchup as a condiment if you’re into that thing, I don’t judge.
	}


\end{recipe}